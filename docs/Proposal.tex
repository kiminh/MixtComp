\documentclass[iutinfo, cadre]{ens-ustl}

%---------------------------------------------------------------------
% Packages pour inclure des images postscipt
\usepackage{epsfig}
%\usepackage[all]{xy}

%---------------------------------------------------------------------
%
%\SetUnicodeOption
\usepackage[utf8]{inputenc}
\usepackage[T1]{fontenc}
%\usepackage[french]{babel}

%---------------------------------------------------------------------
%
\usepackage{amsmath}
\usepackage{amssymb}
\usepackage{theorem}

%---------------------------------------------------------------------
%
\usepackage{algorithm}
\usepackage{algorithmic}

%\usepackage{epsfig}

\usepackage{url}
\urlstyle{sf}

%---------------------------------------------------------------------
%
\renewcommand {\algorithmiccomment}[1]{/*\textit{ #1 }*/}
\renewcommand {\algorithmicrequire}{\textbf{Input:}}
\renewcommand {\algorithmicensure}{\textbf{Output:}}

%---------------------------------------------------------------------
%
\newcommand {\Z}{\mathbb{Z}} %% integers
\newcommand {\Q}{\mathbb{Q}} %% rationals
\newcommand {\C}{\mathbb{C}} %% complex
\newcommand {\D}{\mathbb{D}} %% unit disc
\newcommand {\T}{\mathbb{T}} %% unit circle
\newcommand {\F}{\mathbb{F}} %% Free group
\newcommand {\N}{\mathbb{N}}
\newcommand {\R}{\mathbb{R}}

%---------------------------------------------------------------------
%
\newcommand{\Ker}{\mathrm{Ker}} %% noyau
\renewcommand{\Im}{\mathrm{Im}} %% image

\titre{Specifications for the Mixture Composer Library}
\enseignant{S. Iovleff \\ P. Bathia \\ C. Biernacki}

\begin{document}
\maketitle

\section{Coding conventions}

For the coding we use the java coding convention. All data members
are defined with an underscore at the end of their name, like \verb+data_+.

Data members referring to pointer start with \verb+p_+

All alternatives should be enclosed in \verb+enum+.

The notations proposed in this document can be modified if needed
or enclosed in namespace in order to avoid name collision.


\section{Components of the library}
\subsection{Algorithms}

The algorithms will be:
\begin{verbatim}
enum Algo
{
  em_,
  cem_,
  sem_,
  mcem_,
  exact_
}
\end{verbatim}

The priority is to implement the \verb+*em+ versions, the \verb+exact_+ is just there in case of, but
should not be implemented. Should we add stochastic minimization algorithms ?

\subsection{Stopping criteria}

The stopping criteria will be:
\begin{verbatim}
enum StopCriteria
{
  deltaLnLikelihood_,
  deltaPostProbabilities_,
  deltaParameters_,
  nbIterMax_
}
\end{verbatim}
It should be possible to mix different criteria, for example
\verb+deltaLnLikelihood_|nbIterMax_+ mean that we want to stop the iteration when one
of the criteria is true.

\subsection{Data initialization}

The initialization of the algorithm will be:
\begin{verbatim}
enum Initialization
{
  randomPartition_,
  randomParameters_,
  givenPartition_,
  givenPosteriorProbabiblity_,
  givenParameters_
}
\end{verbatim}

Did i forget a method for initialization ? The random cases are the priority.
In case of heterogeneous distributions it will be difficult to find a way for initialization with
given parameters.

\subsection{Strategies}

A strategy is composed of three parts:
\begin{enumerate}
  \item multiple initializations,
  \item for each initialization a short run,
  \item A long run.
\end{enumerate}

\subsubsection{short runs}

A short run will be an arbitrary number of sequence \verb+(Algo_, StopCriteria)+. For example:
\begin{verbatim}
{(sem_, 1000), (cem_, 0.01|1000)}, {(em, 0.01)}
\end{verbatim}
mean that in a short run, there is 1000 iterations of the SEM algorithm, and at most 1000 iterations
of CEM that will be stopped  if some other criterion have a delta less than 0.01 and finally iterations of the EM
until the delta of some criteria is less than 0.01.

\subsubsection{long run}

A long run is initialized with the better of the short run and a StopCriteria.
If there is no short run, with an initialization.

\subsection{Model Selection Criteria}
The model criteria will be:
\begin{verbatim}
enum Criteria
{
  aic_,
  bic_,
  icl_,
  cv_,
  penExtern_
}
\end{verbatim}
It should be possible to let an user to define its own penalization criteria (\verb+penExtern_+ option).
Should we add cross-validation ?

\section{Plugin specifications (Parmeet) }


\end{document}
