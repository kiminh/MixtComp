\documentclass[a4paper,11pt]{article}
\usepackage[utf8x]{inputenc}
\usepackage{fullpage}
\usepackage{hyperref}
\usepackage{listings}

% Title Page
\title{MixComp Specification: Framework and Developer}

\begin{document}
\maketitle
\tableofcontents

\begin{abstract}
This is a mini-report for global architecture of MixComp. We will make use of two terms
"framework" and "developer". The framework provides a unified development
environment whereas developer will refer to the coding part that will make use of
the framework (without modifying the framework) to realize new mixture laws. In
practice, this is just a separation of Interface (framework) and Implementation (developer). 
\end{abstract}
\section{Framework Introduction}
The idea of unified framework is to integrate the existing and future clustering models.
The things that can be expected from the framework are:
\begin{enumerate}
\item Unified environment for composite mixture model based on the independence assumption (amongst different mixture models).
\item Abstract(Interface) plug-in class that must be derived by developer to develop a
new mixture law.
\item Facilitate creation of Rpackage, Web Interface, GUI and other High level functionalities.
\item Facilitate parallelization using distributed and shared memory models.
\item Take input from user and run the whole software.
\end{enumerate}

The developer is not expected to change the functionalities provided by this framework
but can only provide concrete behavior to these functionalities. Any new functionalities
should be first introduced in the framework. Hence the framework is expected to evolve
with time. The developer is only expected to provide all the low level implementations
needed by the framework without worrying about how these functionalities will be
brought together to realize a composite mixture model. Hence in short, the developer
can concentrate on development of existing and new mixture laws without worrying
about how to run them in integrated(composite) environment. The developer is free
to chose it's development environment (including numerical libraries for example
STK++, Eigen, Lapack or anything more suitable to developer needs) and in no way
will be restricted by framework. For example, a developer can re-factor their
existing codes and fit them into the framework (Quentin existing codes will be
a good test for it) or one can re-implement from scratch  using this framework
(development of simple models including Bernoulli and Gaussian(with diagonal
co-variance matrix) with Serge on STK++ platform will be a good test for it).
This is the most interesting feature of this architecture as it will allow to
independently develop new models.

\section{Plug-in Specification}

The Plug-in specification includes description on how to define a new mixture law using the framework.
To create a new mixture law, the developer 
must create a new class inherited from {\bf IDeveloper} class.
There are certain functions that are already defined in the {\bf IDeveloper} 
class and hence can be called directly inside the inherited class. These functions are {\bf protected} and hence cannot be called from outside (publicly). 
They are enumerated below:
\begin{itemize}
 \item {\bf int nbCluster()}\\
This function will return the number of clusters.
 \item {\bf int nbSample()}\\
This function will return the number of samples.
\item {\bf double** conditionalProbabilities()}\\
This function will return the conditional probabilities. 
\item {\bf int* classLabels()}\\
This function will return the class labels.
\item {\bf double* proportions()}\\
This function will return the class proportions.
\end{itemize}

The above functions must be called whenever you need above variables. This will ensure that all the mixture laws in composite environment should have the same
values for all the above variables.

The {\bf IDeveloper} class also contained following member variables, hence you must not redefine/redeclare them inside your inherited class as you will inherited these members automatically.
\begin{itemize}
\item {\bf int nbVariable\_ : }Number of variables. This value will be set inside the setData() function as explained in section \ref{datahandling}. 
\item {\bf char id\_ : }Identification character. This value will be automatically set by framework. 
\end{itemize}
\subsection{Interface specification}
The {\bf IDeveloper} class contains Interface methods that 
are mostly based on SEM-gibbs algorithm to allow for generalization. Most of these methods are abstract, hence the developer must provide
concrete implementation for them in the derived class. The functions ending with "() = 0" are abstract. The functions ending with "() \{\}" do nothing by default and the functions
ending with "()" are already implemented in framework (the developer can of-course
overwrite them in some cases for performance reasons.) The following interface functions are available:
\begin{itemize}
\item {\bf virtual void initializeStep() = 0} \\
This function must be use for initialization stuffs including initialize of all
the parameters. You should call classLabels() function to get the class labels needed to initialize your mixture law parameters.
This method will be called only once in the very beginning.
\item {\bf virtual IDeveloper* clone() = 0} \\
This is a standard clone function in usual sense. It must provide a new object of your class with values of various parameters equal to the values of calling
object. In other words, this is equivalent to polymorphic copy constructor. 
\item {\bf virtual void imputationStep() \{\}}\\
This function should be used for Imputation of data. The default implementation (in the base class) is to do nothing.
\item {\bf virtual void samplingStep() = 0} \\
This function must be used for simulation of all the latent variables and/or missing data
excluding class labels. The class labels will be simulated by the framework itself because to do so
we have to take into account all the mixture laws. 
\item {\bf virtual void paramUpdateStep() = 0}\\
This function is equivalent to Mstep. This function must be defined by developer to update parameters.
\item {\bf virtual void storeIntermediateResults(int iteration) \{\}}\\
This function should be used to store any intermediate results during various iterations after the burn-in period. The {\bf iteration} argument gives the 
iteration number beginning after the burn-in period.
\item {\bf virtual void finalizeStep() \{\}}\\
This step can be used by developer to finalize any thing. It will be called only once after we
finish running the SEM-gibbs algorithm.
\item {\bf virtual double posteriorProbability(int sample\_num,int Cluster\_num) = 0}\\
This function must be defined by developer to return the posterior probability (PDF) for corresponding sample and cluster.
\item {\bf virtual double logLikelihood() const = 0}\\
This should be defined to return the current log-likelihood.
\item {\bf virtual int freeParameters() const = 0}\\
This should be used to return the number of free parameters.
\item {\bf virtual void setData() = 0}\\
This function must be defined to set the data into your data containers. To facilitate data handling, framework provide templated functions,
that can be called directly to get the data. Data-Handling is explained in more details in section \ref{datahandling}.
\item {\bf virtual void writeParameters(std::ostream\&) const \{\}}\\
This function can be used to write summary of parameters on to the output stream. This is useful for debugging purposes but need not be defined necessarily. The
default implementation is to do nothing.
\end{itemize}

There is no notion of Estep, Mstep, Sstep and Cstep in the plug-in part. These names will be used inside the framework to make
it generic enough to expand for future. 

\subsection{Data Handling}
As briefed in previous part, data must be set inside the virtual function {\bf setData()}. The framework provides templated classes for data handling. The 
psedo code for {\bf setData()} is given below:
\begin{lstlisting}
virtual void setData()
{
  Data<type> mydatahandler;
  type** data = mydatahandler.getData(id_,nbVariable_);
  // Fill up your data container using data pointer
}
\end{lstlisting}
You should replace "type" in above pseudo code with you own data-type which can be {"double", "int", "bool" or "string"}. You don't have to worry about the 
{\bf id\_} variable as it will be set by framework. Also this function will set the variable {\bf nbVariable\_} as it is passed by reference.

\label{datahandling}
\section{stk++ Statistical Models}

{\bf TODO}

\appendix

\section{Coding conventions}

For the coding we use the java coding convention. All data members
are defined with an underscore at the end of their name, like \verb+data_+.
Data members referring to pointer start with \verb+p_+.
Data members that are vector should start with \verb+v_+ and that are matrices should begin with \verb+m_+.
All alternatives should be enclosed in \verb+enum+.
The notations proposed in this document can be modified if needed
or enclosed in namespace in order to avoid name collision.


\section{Components behaviors}
\subsection{Algorithms}

The currently available algorithms are:

\begin{enumerate}
 \item {\bf semgibbs\_:} This perform operations based on SEMGibbs algorithm and is run on model (passed as argument) as shown in the following pseudo code.
\end{enumerate}

\subsection{Strategy}

The currently available strategies are:
\begin{enumerate}
 \item {\bf iterations\_:} This is a simple strategy based on number of iterations. The algorithm will be ran on model for given maximum number of iterations. This will be repeated for given number of times (tries) and the best model will be selected at the end based on log-likelihood.
\end{enumerate}

\subsection{Model Selection Criteria}
The model selection criteria are:
\begin{verbatim}
enum Criteria
{
  aic_,
  bic_,
  icl_
}
\end{verbatim}

\end{document}
